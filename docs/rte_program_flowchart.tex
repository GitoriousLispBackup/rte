\documentclass[landscape]{article}
\usepackage[margin=1in]{geometry}

%\usepackage[latin1]{inputenc}
\usepackage{tikz}
\usetikzlibrary{shapes,arrows}
\usepackage{verbatim}

\begin{comment}

:Title: Simple flow chart
:Tags: Diagrams

With PGF/TikZ you can draw flow charts with relative ease. This flow chart from [1]_ outlines an algorithm for identifying the parameters of an autonomous underwater vehicle model. 

Note that relative node placement has been used to avoid placing nodes explicitly. This feature was introduced in PGF/TikZ >= 1.09.

.. [1] Bossley, K.; Brown, M. & Harris, C. Neurofuzzy identification of an autonomous underwater vehicle `International Journal of Systems Science`, 1999, 30, 901-913 

\end{comment}

\begin{document}
\pagestyle{empty}

% Define block styles
\tikzstyle{decision} = [diamond, draw, fill=blue!20, 
    text width=4.5em, text badly centered, node distance=3cm, inner sep=0pt]

\tikzstyle{block} = [rectangle, draw, fill=blue!20, 
    text width=5em, text centered, rounded corners, minimum height=4em]
\tikzstyle{subblock} = [rectangle, draw, fill=green!20, 
    text width=5em, text centered, rounded corners, minimum height=4em]

\tikzstyle{line} = [draw, -latex']

\tikzstyle{cloud} = [draw, ellipse,fill=red!20, node distance=3cm,
    minimum height=2em]

\begin{tikzpicture}[node distance = 1.25in, auto]
  % Place nodes
  %%%% Headline flowchart
  \node [cloud] (dataset) {data set};
  \node [block, below of=dataset] (extract) {extract t-h pair};
  \node [block, right of=extract] (normalize) {normalize input};
  \node [decision, right of=normalize, node distance=2in] (parse) {parse};
  \node [decision, right of=parse, node distance=3.5in] (text) {text?};
  \node [block, above right of=text] (assert) {assert all};
  \node [block, below right of=text] (question) {question h};
  \node [block, above right of=question] (return) {return entailing sentences};
  
  %%%% How to normalize input
  \node [subblock, above of=normalize, node distance=1in] (split) {split t into sentences};
  \node [subblock, above of=split, node distance=1in] (vocab) {add unknown words to lexicon};

  %%%% How to parse input
  \node [subblock, below of=parse, node distance=1in] (dependency) %
                                   {stanford $\rightarrow$ dependency};
  \node [subblock, left of=dependency] (lkb) {lkb $\rightarrow$ mrs};
  \node [subblock, right of=dependency] (tree) %
                                         {stanford $\rightarrow$ tree};
  \node [subblock, below of=dependency, node distance=1in] (entities) %
                                        {extract entities and align with KB};
  \node [subblock, below of=entities] (sfy) %
                                      {nativize the linguistic representation};

  % Draw edges
  %%%% Headline flowchart
  \path [line] (dataset) -- (extract);
  \path [line] (extract) -- (normalize);
  \path [line] (text) -- node {yes} (assert);
  \path [line] (text) -- node {no}(question);
  \path [line] (assert) -| (parse);
  \path [line] (question) -- (return);

  %%%% How to normalize input
  %%\path [line] (normalize) -- (parse);
  \path [line] (normalize) -- (split);
  \path [line] (split) -- (vocab);
  \path [line] (vocab) -- (parse);
  
  %%%% How to parse input
  %%\path [line] (parse) -- (text);
  \path [line] (sfy) -| (text);
  \path [line] (parse) -- (lkb);
  \path [line] (parse) -- (dependency);
  \path [line] (parse) -- (tree);
  \path [line] (lkb) -- (entities);
  \path [line] (dependency) -- (entities);
  \path [line] (tree) -- (entities);
  \path [line] (entities) -- (sfy);

\end{tikzpicture}

\end{document}

